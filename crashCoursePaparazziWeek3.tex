\documentclass{article}

\usepackage[margin=3cm]{geometry}

\usepackage{fancyhdr}
\pagestyle{fancy}
\usepackage{graphicx}
\usepackage{color}
\usepackage{xcolor}
\definecolor{code}{rgb}{.92,.92,.99}
\definecolor{codered}{rgb}{.92,.62,.62}
\definecolor{darkgreen}{rgb}{.2,.62,.18}
\usepackage{listings}
\lstset{xrightmargin=10pt,xleftmargin=10pt,language=Java,captionpos=b,tabsize=3,frame=none,keywordstyle=\color{blue},commentstyle=\color{darkgreen},stringstyle=\color{red},showstringspaces=false,basicstyle=\footnotesize\ttfamily,emph={label},backgroundcolor=\color{code}}
%%numbers=left,numberstyle=\tiny,numbersep=5pt,breaklines=true,\

\newenvironment{code}
{\begin{minipage}[l]{\textwidth}}
{\end{minipage}}

\newcommand{\ccpmaketitle}[3] {
	\fancyhead[LO,LE]{Crash Course Paparazzi 2016}
	\fancyfoot[LO,LE]{{\small introPaparazzi}}
	\fancyfoot[CO,CE]{\thepage}
	\author{Roland Meertens, Christophe de Wagter, Guido de Croon}
	\title{\bf Crash Course Paparazzi 2016\\#1\\{\large #3#2}}
	\date{Februari 2016}
	\setlength{\parindent}{0em}
	\maketitle
}


\begin{document}
\ccpmaketitle{Creating code}{\ldots Making decisions}{Lesson 3}

\subsection*{Introduction}
Last week you could to let your drone follow a flight plan. You were manually placing your waypoints, and manually selecting what block to fly. This week will spice things up by letting the drone decide what to do. 
To do this we will try writing code, but not before more explanations of Paparazzi are given. 
\subsection*{Goals of this week}
\begin{itemize}
\item Know how and where to get help
\item Adding a module
\item Learning how to debug
\item Making decisions onboard
\item Moving waypoints
\end{itemize}
\subsection*{How to get help}
As you start working with the Paparazzi autopilot you will get loads of questions. Paparazzi is written in the C language, and it can be a big hurdle to start programming in this language. I recommend reading the book "Head First C". Although the book is full with strange images, the explanations are very clear. 

Here is where you can get help if you are stuck: 
\begin{itemize}
\item If you can't find something in Paparazzi use the "Find in file" function of Eclipse. This will help you find the files or variables you are searching for.
\item If you can't find it with Eclipse you can try searching the Paparazzi wiki. It is a huge collection of information and almost everything is documented there. If you find something that is not yet documented: please add it yourself to the wiki and help your fellow students. 
\item If you still can't find a specific Paparazzi function chat with the developers of Paparazzi on gitter.im/paparazzi.
\item If you have questions about the C language: check Stackoverflow for answers to your question.
\item You can also ask questions on the course's forum. 
\end{itemize}

\subsection*{Adding a module}
A tutorial on how to add a module to your program can be found here: \url{http://wiki.paparazziuav.org/wiki/Modules}.
First check that the module you want to have does not exist already. Be sure to check the create module tool that is already available to you in Paparazzi (paparazzi/create\_module.py). 
Create a module now, add it to your airframe file, and recompile everything. 

\subsection*{Learning how to debug}
When programming your drone you could encounter that the drone does not do what you expected from the first time. In that case there are two ways to check what the values of variables on your drone are:
\begin{itemize}
	\item Paparazzi messages
	\item Printing to the terminal
\end{itemize}
To learn more about paparazzi messages check this webpage: \url{http://wiki.paparazziuav.org/wiki/Telemetry}
Create a new message now and send an arbitrary value. Check if you receive it by running the messages program in Paparazzi. 
As for printing to the terminal: you can use the following in your code:

\begin{verbatim}
printf("Integer variable: %d\n",myIntegerVariable);
\end{verbatim}
To see this text appear open a terminal on your laptop and type:
\begin{verbatim}
telnet 192.168.42.1
cd data/ftp/internal_000
killall -9 ap.elf 
./ap.elf
\end{verbatim}
Your program is now restarted, and your message will appear in your terminal. 

\subsection*{Making decisions onboard}
As you saw last week you can leave blocks in your flight plan by adding an exception. Paparazzi allows you to use a variable of your module as a trigger for an exception. When the variable you selected is not equal to zero, the exception will occur and a next block is selected. 
Let's try to create the following behaviour: 
\begin{itemize}
\item Start at waypoint one, and enter your newly created block. 
\item The drone will now fly towards waypoint two. 
\item As soon as the drone spent five seconds flying towards this waypoint (or was there for too long) we fly back to waypoint one. 
\end{itemize}
Checking if the drone spent a long time in a block can be done in the block itself with the line:
\begin{verbatim}
<exception "gt(block_time,5)" /> 
\end{verbatim}
However, we will create this as a function in your module.h file as such:
\begin{verbatim}
extern void spentLongTimeInBlock(int timeInBlock){
	return timeInBlock > 5;
}
\end{verbatim}
And add it to our block:
\begin{verbatim}
<exception spentLongTimeInBlock(block_time)/>
\end{verbatim}
Fly and verify that your program works. Now you know that you can add complex functions in your module, and call them from your flightplan. 
You are now also able to create exceptions yourself!

\subsection*{Exercise: moving waypoints}
Now that we can make decisions we can also move waypoints. The goal of this exercise is to fly in the pattern of a house inside the arena. 
We do this by flying to a waypoint (WP\_GOAL) and when we reach this waypoint we move the waypoint to a new place. 

\begin{enumerate}
\item Start by creating a function that takes a waypoint as input (note: the waypoint variables can be found in firmwares/rotorcraft/navigation.h. Note that if you want or need to find out how the flightplan xml is converted into c-code, you can find that in var/aircrafts/bebop/generated/flight\_plan.h).
\item Look at the functions that are already in Paparazzi that do something with waypoints in the file sw/airborne/firmwares/rotorcraft/navigation.h. 
\end{enumerate}

Now you should be able to set a waypoint in one corner of the arena(away from the net), and move this waypoint in the shape of a square or house by placing it at different points in the arena as soon as you reach them. 

You can either move a waypoint relative to its old position, or you can move it relative to your current position and heading. To do the latter you should make a placeRelative function that takes a waypoint and a relative move as input and moves the waypoints taking the yaw of the drone into account. This function will surely be of much use in the next few weeks. If you choose to also change your heading while flying this pattern, (find function for it in navigation.h) then take into account that after a significant yaw angle command, drones may need some time to execute it and might move laterally in case of asymmetry or calibration issues. Always be careful not to make brutal yaw changes close to obstacles.

\end{document}

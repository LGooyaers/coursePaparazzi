\documentclass{article}

\usepackage[margin=3cm]{geometry}

\usepackage{fancyhdr}
\pagestyle{fancy}
\usepackage{graphicx}
\usepackage{color}
\usepackage{xcolor}
\definecolor{code}{rgb}{.92,.92,.99}
\definecolor{codered}{rgb}{.92,.62,.62}
\definecolor{darkgreen}{rgb}{.2,.62,.18}
\usepackage{listings}
\lstset{xrightmargin=10pt,xleftmargin=10pt,language=Java,captionpos=b,tabsize=3,frame=none,keywordstyle=\color{blue},commentstyle=\color{darkgreen},stringstyle=\color{red},showstringspaces=false,basicstyle=\footnotesize\ttfamily,emph={label},backgroundcolor=\color{code}}
%%numbers=left,numberstyle=\tiny,numbersep=5pt,breaklines=true,\

\newenvironment{code}
{\begin{minipage}[l]{\textwidth}}
{\end{minipage}}

\newcommand{\ccpmaketitle}[3] {
	\fancyhead[LO,LE]{Crash Course Paparazzi 2016}
	\fancyfoot[LO,LE]{{\small introPaparazzi}}
	\fancyfoot[CO,CE]{\thepage}
	\author{Roland Meertens, Christophe de Wagter, Guido de Croon}
	\title{\bf Crash Course Paparazzi 2016\\#1\\{\large #3#2}}
	\date{Februari 2016}
	\setlength{\parindent}{0em}
	\maketitle
}


\begin{document}
\ccpmaketitle{Creating code}{\ldots Making decisions}{Lesson 3}

\subsection*{Introduction}
Last week you were able to let your drone follow a flightplan. Unfortunately you were manually placing your waypoints, and manually selecting what block to fly. This week we are going to spice things up by letting the drone decide what to do. 

\subsection*{Goals of this week}
\begin{itemize}
\item Know how and where to get help
\item Adding a module
\item Learning how to debug
\item Making decisions onboard
\item Moving waypoints
\end{itemize}
\subsection*{How to get help}
You are now working with a very good tool: the paparazzi autopilot. This autopilot is written in the C language. If you are not a good programmer yet it can be a big hurdle to suddenly start programming your own programs with Paparazzi. Personally I would like to recommend the book "Head first C" (you might be able to find it on the internet). The book is full with strange images and explanations, but don't let this stop you from reading the book. I personally found that it was easier to learn C with a book full of wacky images than with a very dry book. 

If you can't find something in Paparazzi use the "Find in file" function of Eclipse. This will help you find the files or variables you are searching for.
If you can't find it with Eclipse you can try searching the Paparazzi wiki. Almost everything is documented there. 
If you still can't find a specific paparazzi function chat with the developers of Paparazzi on gitter.im/paparazzi.
If you have any questions about the C language: everybody has. Please check stack overflow for answers to your question before posting it to blackboard.

\subsection*{Adding a module}
A tutorial on how to add a module to your program can be found here: XXX

\subsection*{Learning how to debug}
When programming your drone you will often find that the drone does not do what you want. In that case there are two ways to check what the values of variables on your drone are:
\begin{itemize}
	\item Paparazzi messages
	\item Printing to the terminal
\end{itemize}
To learn more about paparazzi messages check this webpage: XXXX
As for printing to the terminal: you can use the following in your code:
printf("Integer variable: %d\n",myIntegerVariable);
To see this text appear open a terminal on your laptop and type:
telnet 192.168.1.1
cd data/video/paparazzi
%killall -9 ap.elf && ./ap.elf

Your program is now restarted, and your printf message will appear. 

\subsection*{Making decisions onboard}
As you saw last week you can leave blocks by adding an exception that evaluates to true. 
Let's try to create the following behaviour: 
\begin{itemize}
\item Start at waypoint one, and enter your newly created block. 
\item The drone will now fly towards waypoint two. 
\item As soon as the drone spent five seconds flying towards this waypoint (or was there for too long) we fly back to waypoint one. 
\end{itemize}
Checking if the drone spent a long time in a block can be dont in the block itself with the line:
XXXXX > 5
However, we are going to create this as a function in your module.h file as such:
extern void spentLongTimeInBlock(int timeInBlock){
	return timeInBlock > 5;
}
And add it to our block as such:
%<exception spentLongTimeInBlock(XXXX)/>

Fly and verify that your program works. Now you know that you can create functions and exceptions yourself!

\subsection*{Moving waypoints}
Now that we can make decisions we can also move waypoints. The goal of this exercise is to fly in the pattern of a house inside the arena. 
%We do this by flying to a waypoint (WP_GOAL) and when we reach this waypoint we move the waypoint to a new place. 
\textbf{Exercise: create a flightplan that performs this behaviour with two blocks}

\begin{enumerate}
\item Start by creating a function that takes a waypoint as input (note: the waypoint variables can be found in XXX).
\item Look at the following functions that are already in Paparazzi that do something with waypoints: XXX, XXX, XXX and XXX. 
\end{enumerate}

Now you should be able to set a waypoint in one corner of the arena, and move this waypoint in the shape of a quare or house by placing it at different points in the arena.

You can either move a waypoint relative to its old position, or you can move it relative to your current position and heading. Now try to do the same by placing your waypoints relative to the drone. As an exercise you should make a placeRelative function that takes a waypoint and a relative move as input and moves the waypoints taking the yaw of the drone into account. 

\end{document}
